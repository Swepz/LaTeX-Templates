\usepackage[scale=2]
{ccicons}
\usepackage{pgfplots}
\usepgfplotslibrary{dateplot}
\usepackage[utf8]{inputenc}
\usepackage[T1]{fontenc}
\usepackage[swedish,english]{babel}
\makeatletter
% pdfusetitle exists class
\hypersetup{allcolors=black, hidelinks}
\makeatother
\usepackage{mathtools}
\usepackage{xcolor}
\usepackage{booktabs}
\usepackage{bookmark}
\usepackage{graphicx}
\usepackage{url}
\usepackage{float}
\usepackage{textpos}
\usepackage{cancel}
\usepackage{makecell}
\usepackage{empheq}  % boxing equations
\usepackage{multicol}
\usepackage{animate}
\usepackage{algorithm}
\usepackage{algorithmic}
\usepackage[b]{esvect} % for the \vv
\usepackage[most]{tcolorbox}
\usepackage{epsfig,graphicx,psfrag}
\usepackage{wrapfig}
\usepackage{times}
% \usepackage{minted}    % Mac OS X: sudo easy_install Pygments
% latex -shell-escape slides.tex

% Define custom colors
\definecolor{harvardcrimson}{rgb}{0.79,0.0,0.09}
\definecolor{cornellred}{rgb}{0.8,0.11,0.11}
\definecolor{bostonuniversityred}{rgb}{0.8,0,0}
\definecolor{burgundy}{rgb}{0.5,0,0.13}
\definecolor{crimsonglory}{rgb}{0.75,0,0.2}
\definecolor{cyanproc}{rgb}{0,0.72,0.92}
\definecolor{denim}{rgb}{0.08,0.38,0.74}

% Declare Math Operators
\DeclareMathOperator{\sign}{sgn}
\DeclareMathOperator{\tr}{tr}
\DeclareMathOperator{\diag}{diag}
\DeclareMathOperator{\rank}{rank}
\DeclareMathOperator{\de}{\,\mathrm{d}\!}
\DeclareMathOperator{\imp}{imp}
\DeclareMathOperator{\stp}{step}
\DeclareMathOperator{\ram}{ram}
\DeclareMathOperator{\dB}{dB}
\DeclareMathOperator{\atan}{atan}
\DeclareMathOperator*{\argmin}{arg\,min}
\DeclareMathOperator*{\argmax}{arg\,max}
\DeclareMathOperator{\nullSpace}{ker}
\DeclareMathOperator*{\minimize}{minimize}
\DeclareMathOperator*{\maximize}{maximize}

% New commands
\newcommand*\blue[1]{%
    \colorbox{blue!10}{\hspace{1em}#1\hspace{1em}}}
\newcommand*\red[1]{%
    \colorbox{red!10}{\hspace{1em}#1\hspace{1em}}}
\newcommand*\green[1]{%
    \colorbox{green!10}{\hspace{1em}#1\hspace{1em}}}
\newcommand*{\myPause}{\pause}
\newcommand*{\laplace}[1]{\mathcal{L}\left\lbrace #1 \right\rbrace(s)}
\newcommand*{\invlaplace}[1]{\mathcal{L}^{-1}\left\lbrace #1 \right\rbrace(t)}
\DeclareMathOperator{\cov}{covar}
\newcommand*{\bs}[1]{\boldsymbol{\mathrm{#1}}}
\newcommand*{\tb}[1]{\alert{\textbf{#1}}}
\newcommand*{\colr}[1]{\textcolor{red}{\textbf{#1}}}
\newcommand*{\colb}[1]{\textcolor{blue}{\textbf{#1}}}
\newcommand*{\colt}[1]{\textcolor{teal}{\textbf{#1}}}

\newtheorem*{proposition}{Proposition}
\newtheorem*{remark}{Remark}

%% Beamer slides settings

\setbeamercolor{frametitle}{bg=white, fg=harvardcrimson}
\setcounter{tocdepth}{2} % Only sections and subsections in toc

\setbeamercolor{alerted text}{fg=harvardcrimson,bg=white}
\setbeamertemplate{theorems}[unnumbered]
\setbeamertemplate{theorems}[ams style]
\setbeamertemplate{theorems}[numbered]

\usetheme[outer/progressbar=foot,]{metropolis}
\metroset{sectionpage=progressbar}
\metroset{subsectionpage=progressbar}
\setbeamercolor{background canvas}{bg=white}
\metroset{block=fill}

\setbeamertemplate{itemize items}[circle]
\setbeamertemplate{itemize subitem}[square] % Beamer-Template/-Color/-Font
\setbeamertemplate{itemize subsubitem}[ball] % Beamer-Template/-Color/-Font


\graphicspath{{image/}}
\titlegraphic{\hfill\includegraphics[width=30mm]{__logo.eps}}


\setbeamertemplate{frame footer}{
    \hspace*{1em}\vspace*{-2mm}\tikz{\node[font=\tiny]{\insertsectionhead};}%
}
\makeatletter
\setbeamertemplate{title page}{
  \vbox{}
  \vfill
  \begingroup
    \begin{minipage}{.75\textwidth}
    \begin{beamercolorbox}[sep=8pt]{white}
      \usebeamerfont{title}\inserttitle\par%
      \ifx\insertsubtitle\@empty%
      \else%
        \vskip0.25em%
        {\usebeamerfont{subtitle}\usebeamercolor[fg]{subtitle}\insertsubtitle\par}%
      \fi%     
    \end{beamercolorbox}%
    \end{minipage}
    \begin{minipage}{.2\textwidth}
        {\usebeamercolor[fg]{titlegraphic}\inserttitlegraphic\par}
    \end{minipage}
    {\centering\vspace*{-1em}\rule{0.95\textwidth}{1pt}}
    \vskip1em\par
    \begin{minipage}{0.7\textwidth}
    \begin{beamercolorbox}[sep=8pt]{white}
      \usebeamerfont{author}\insertauthor
    \end{beamercolorbox}
    \end{minipage}

    \begin{minipage}{.5\textwidth}
    \begin{beamercolorbox}[sep=8pt]{white}
      \usebeamerfont{institute}\insertinstitute
    \end{beamercolorbox}
    \end{minipage}

  \endgroup
  \vfill
}
\makeatother
